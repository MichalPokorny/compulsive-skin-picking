\documentclass[a4paper]{article}
% \usepackage{fullpage}
\usepackage[utf8]{inputenc}
\usepackage[nottoc,numbib]{tocbibind}
% \usepackage{amssymb}
% \usepackage{amsmath}
\usepackage[czech]{babel}
\usepackage{listings}
\usepackage{todonotes}
\usepackage{fancyhdr}
\usepackage{listings}
\usepackage{wrapfig}
\usepackage{caption}
\usepackage{subcaption}
\usepackage{placeins}

% \definecolor{string}{rgb}{0.7,0.0,0.0}
% \definecolor{comment}{rgb}{0.13,0.54,0.13}
% \definecolor{keyword}{rgb}{0.0,0.0,1.0}
% \lstset{
% % numbers=left,
%   language=Ruby,
%   frame=single,
%   tabsize=2,
%   basicstyle=\footnotesize\ttfamily,
%   keywordstyle=\color{keyword}\textbf,
%   commentstyle=\color{comment}\textit,
%   stringstyle=\color{string}
% }
% \pgfdeclarelayer{background}
% \pgfdeclarelayer{foreground}
% \pgfsetlayers{background,main,foreground}

\begin{document}
\title{Programátorská dokumentace \\ CompulsiveSkinPicking - C\# CSP solver}
\author{Michael Pokorný}
\date{14.7.2014}

\maketitle

\tableofcontents

\section{Reference}

\subsection{\texttt{Debug}}
Třída \texttt{Debug} slouží pro snadné ladění knihovny. Obsahuje metody
\texttt{Write} a \texttt{WriteLine} s API analogickým k \texttt{System.Console}.
Ladění lze zapínat a vypínat pomocí statické boolean proměnné
\texttt{Debug.doDebug}. Je-li ladění zapnuto, chovají se zmíněné metody jako
přislušné metody \texttt{System.Console}. V opačném připadě neprovádí tyto
metody nic.

\subsection{\texttt{IExternalEnumerator}}
Generické rozhraní \texttt{IExternalEnumerator<T>} reprezentuje "ukazatel
dovnitř seznamu", který lze "předávat hodnotou".
Vyžaduje implementaci vlastnosti \texttt{T Value \{ get; \}}, která umožňuje
přečíst prvek, na který tento "ukazatel" právě ukazuje, a implementaci
metody \texttt{bool TryProgress(out IExternalEnumerator<T>)}. Tato metoda se
pokusí vrátit "ukazatel" na další prvek v seznamu, pokud existuje. Pokud žádný
další prvek neexistuje, vrátí \texttt{false}.

\subsection{\texttt{Extensions}}
Statická třída \texttt{Extensions} přidává dvě extension metody třídě
\texttt{List<T>}. Metoda \texttt{GetExternalEnumerator} vrátí
\texttt{IExternalEnumerator} přes prvky tohoto seznamu.
\texttt{GetTransformedExternalEnumerator} umožňuje navíc předat funkci, která
překládá z prvků seznamu na hodnoty vracené externím enumerátorem.

\subsection{Reprezentace hodnot, \texttt{ValueRange}}
Hodnoty proměnných jsou reprezentovány typem \texttt{int}, z čehož plynou
omezení na jejich rozsah. Struktura \texttt{ValueRange} reprezentuje interval
hodnot.

Obsahuje dvě veřejné položky: \texttt{Minimum} a \texttt{Maximum}. Interval
je daný jako množina těch \texttt{x}, že \texttt{Minimum <= x < Maximum}.

Tato třída má dva konstruktory. Jeden přijímá pouze jeden \texttt{int} a vytvoří
\texttt{ValueRange} reprezentující interval obsahující pouze tuto jednu doménu.
Druhý přijímá hodnotu pro \texttt{Minimum} a pro \texttt{Maximum}.

Metoda \texttt{SplitAndRemove} se volá, když je potřeba z intervalu odebrat
prvek, který dostane jako argument. Vrací \texttt{IEnumerable<ValueRange>},
který enumeruje přes všechny
intervaly, které vzniknou, když se z intervalu odebere prvek. Například volání
\texttt{SplitAndRemove} na intervalu $\{1,2,3,4\}$ s parametrem 2 vrátí
intervaly $\{1\}$ a $\{3,4\}$.

Metoda \texttt{Contains} přijímá jeden \texttt{int} parametr a vrací, je-li
tento parametr obsažen v intervalu.

Vlastnost \texttt{IsSingleton} vrací, jedná-li se o interval s jediným prvkem.
Pokud se jedná o interval s jediným prvkem, lze tento prvek přečíst vlastností
\texttt{Singleton}.

Třída \texttt{ValueRange} přetěžuje metodu \texttt{ToString} tak, že vrací
lidsky čitelnou reprezentaci intervalu (například \texttt{"VR<10>"} nebo
\texttt{"VR<1...3>"} pro $\{1,2\}$).

Vlastnost \texttt{Size} vrací počet hodnot uvnitř intervalu.
Kromě toho příbuzná metoda \texttt{AtLeastElements} s integerovým parametrem \texttt{x} vrací
\texttt{true}, pokud má interval aspoň \texttt{x} prvků. Jedná se o metodu
oddělenou od \texttt{Size}, protože první verze \texttt{ValueRange} obsahovala
podporu pro nekonečné intervaly, jejichž velikost nelze reprezentovat integerem.

Interval lze indexovat integerem:
\begin{lstlisting}
var range = new ValueRange(5, 10);
Console.WriteLine(range[3]); // 8
\end{lstlisting}

Nakonec statická vlastnost \texttt{ValueRange.Boolean} vrací interval příslušící
booleovským proměnným (tedy $\{0;1\}$).

\subsection{\texttt{Variable}}
Třída \texttt{Variable} reprezentuje CSP proměnnou. Je vázána na konkrétní
instanci třídy \texttt{Problem}. V rámci tohoto problému má veřejný unikátní
identifikátor \texttt{Identifier} typu \texttt{string}. Dále obsahuje vlastnost
\texttt{Range} typu \texttt{ValueRange}, která ukládá interval, ve kterém se
musí nacházet hodnota této proměnné. Tento interval musí vždy být konečný.

Třída \texttt{Variable} přetěžuje metodu \texttt{ToString} a vrací
\texttt{"Var<(Identifier)>"}. Dále obsahuje metodu \texttt{CompareTo}, která
očekává jako parametr jinou proměnnou a vrací porovnání identifikátorů těchto
dvou proměnných.

\subsection{\texttt{IBacktrackable}}
Rozhraní \texttt{IBacktrackable<T>} je společné pro všechny třídy, které jsou
verzovatelné -- je možné uložit jejich stav do "úložného bodu" a později je
možné tyto třídy požádat, aby se vrátily do předchozího "úložného bodu".
Toto lze implementovat například \texttt{Stack}em ukládajícím vnitřní stav
třídy, ale v některých případech toto lze provádět efektivněji. Tento interface
odstiňuje detaily ukládání předchozích stavů od uživatelů těchto tříd.

Tento interface používá "idiom rekurzivní šablony" -- očekávané použití je
následující: \texttt{class MyClass: IBacktrackable<MyClass>}.

Vyžaduje implementaci následujících metod:
\begin{itemize}
\item \texttt{T DeepDuplicate()}: vrátí hlubokou kopii třídy, která bude mít
	separátní historii.
\item \texttt{void AddSavepoint()}: uloží "do stacku" stav této třídy
\item \texttt{void RollbackToSavepoint()}: vrátí třídu do posledního stavu "na
	stacku" a "odebere tento stav ze stacku"
\item \texttt{bool CanRollbackToSavepoint \{ get; \}}: vrací, lze-li
	třídu vrátit do nějaké předchozí verze
\end{itemize}

\subsection{Částečné přiřazení hodnot: \texttt{IVariableAssignment}}
Rozhraní \texttt{IVariableAssignment} reprezentuje částečné přiřazení hodnot
proměnným -- každá proměnná může buď mít přiřazenou hodnotu, nebo může mít danou
množinu hodnot, které nejsou vyloučené. Třídy implementující
\texttt{IVariableAssignment} musí mít metodu \texttt{Dump}, která vypíše lidsky
čitelné přiřazení hodnot na standardní výstup, vlastnost \texttt{Variables},
která vrátí \texttt{IEnumerable<Variable>} enumerující všechny proměnné a
indexer, který přijímá parametr typu \texttt{Variable} a vrací instanci
\texttt{IVariableManipulator}. Tento indexer slouží jako rozhraní k manipulaci
s hodnotami nebo doménami jednotlivých proměnných.

Protože instance \texttt{IVariableAssignment} jsou část \texttt{SolutionState},
což je třída, přes kterou solver provádí DFS, musí implementace
\texttt{IVariableAssignment} implementovat
\texttt{IBacktrackable<IVariableAssignment>}.

\texttt{IVariableManipulator} vyžaduje následující vlastnosti a metody:
\begin{itemize}
\item
	\texttt{void Restrict(int v)}: odebere z definičního oboru této proměnné
	hodnotu \texttt{v}
\item
	\texttt{int Value \{ get; set; \}}: getter umožňuje číst hodnotu
	proměnné, je-li už plně přiřazena. Setter proměnné přiřadí definiční
	obor obsahující pouze danou hodnotu, neboli tuto proměnnou na tuto
	hodnotu nastaví.
\item
	\texttt{bool Ground \{ get; \}}: vrací, má-li tato proměnná jedinou
	hodnotu.
\item
	\texttt{bool HasPossibleValues \{ get; \}}: vrací, je-li definiční obor
	této proměnné neprázdný.
\item
	\texttt{bool CanBe(int v)}: pro danou hodnotu řekne, je-li obsažena v
	současném definičním oboru proměnné.
\item
	\texttt{IExternalEnumerator<int> EnumeratePossibleValues()}: vrátí
	externí enumerátor možných hodnot proměnné.
\item
	\texttt{int PossibleValueCount \{ get; \}}: vrací počet možných hodnot
	proměnné.
\end{itemize}

\subsubsection{\texttt{VariableAssignment}}
\texttt{VariableAssignment} je konkrétní implementace \texttt{IVariableAssignment}.
Domény reprezentuje instancemi \texttt{Domain}, tedy seznamy intervalů.

\subsection{\texttt{Constrains.IConstrain}}
CSP podmínky jsou reprezentovány jako instance rozhraní
\texttt{Constrains.IConstrain}. Toto rozhraní vyžaduje implementaci
následujících metod:
\begin{itemize}
\item
\texttt{IEnumerable<ConstrainResult> Propagate(\\
IVariableAssignment, IEnumerable<PropagationTrigger>)} \\
	Tato metoda dostane částečné přiřazení hodnot a seznam změn, které
	proběhly při poslední propagaci podmínek a vrátí nové informace, které
	podmínka umí odvodit.
\item
	\texttt{bool Satisfied(IVariableAssignment)}: vrátí \texttt{true}, pokud
	při daném přiřazení proměnných je podmínka již splněna; jinak vrátí
	\texttt{false}.
\item
	\texttt{IEnumerable<Variable> Dependencies \{ get; \}}: tato vlastnost
	vrací seznam proměnných, na kterých závisí splněnost této podmínky.
	Tento seznam musí být konstantní v průběhu řešení problému.
\end{itemize}

\subsection{\texttt{ConstrainResult}}
Struktura \texttt{ConstrainResult} reprezentuje informaci, kterou podmínka umí
odvodit podle daného dosazení do proměnných a nových informací z propagace
jiných podmínek. Enumerace \texttt{ConstrainResult.Type} určuje následující typy
struktury \texttt{ConstrainResult}:
\begin{itemize}
\item
	\texttt{Success}: podmínka byla splněna a žádné další dosazení toto
	nezmění. Solver bude odteď v této větvi tuto podmínku ignorovat, protože
	už nemůže být porušena.

	Tento výsledek lze získat statickou vlastností
	\texttt{ConstrainResult.Success}.
\item
	\texttt{Failure}: podmínka již nemůže nikdy být splněna. Solver se
	pokusí o backtrack, protože současné částečné řešení už nemůže vést na
	úplné řešení.

	Tento výsledek lze získat statickou vlastností
	\texttt{ConstrainResult.Failure}.
\item
	\texttt{Restrict}: bylo odvozeno, že proměnná \texttt{X} nemůže mít
	hodnotu \texttt{x}. Lze získat statickou metodou
	\texttt{ConstrainResult.Restrict(X, v)}.
\item
	\texttt{Assign}: bylo odvozeno, že proměnná \texttt{X} musí ke splnění
	podmínky mít hodnotu \texttt{x}. Lze získat statickou metodou
	\texttt{ConstrainResult.Assign(X, v)}.
\end{itemize}
\texttt{ConstrainResult} přetěžuje \texttt{ToString} tak, aby vracelo lidsky
čitelnou reprezentaci. Veřejné vlastnosti \texttt{Type type}, \texttt{int value}
a \texttt{Variable variable} umožňují nahlížet dovnitř instance.

\subsection{\texttt{Domain}}
Třída \texttt{Domain} reprezentuje seznam hodnot, jaké může nabývat nějaká CSP
proměnná. Implementuje \texttt{IBacktrackable<Domain>}.
Má bezparametrický konstruktor a konstruktor přijímající \texttt{ValueRange},
který bude sloužit jako obsah vytvořené domény.

Doména je interně reprezentovaná jako uspořádaný seznam intervalů. Pro rychlejší
řešení by šlo implementovat podobné třídy ukládající hodnoty například v malém
poli.

Dále má následující veřejné metody a vlastnosti:
\begin{itemize}
\item
	\texttt{int Size \{ get; \}}: vrací počet hodnot v doméně.
\item	\texttt{bool Ground \{ get; \}}: vrací, má-li doména jediný prvek.
\item	\texttt{bool IsEmpty \{ get; \}}: vrací, je-li doména prázdná.
\item	\texttt{bool Contains(int value)}: ptá se domény, obsahuje-li nějaký
	integer
\item	\texttt{void Restrict(int value)}: odebere danou hodnotu z domény
\item	\texttt{int Value \{ get; set; \}}: má-li doména jedinou hodnotu,
	getter ji vrátí. Setter tuto doménu nastaví na příslušnou hodnotu (pokud
	ji doména obsahuje).
\item	\texttt{override string ToString()}
\end{itemize}

\subsection{\texttt{ObjectiveDirection}}
Enumerace \texttt{ObjectiveDirection} reprezentuje dva možné směry CSP
optimalizace: \texttt{ObjectiveDirection.Minimize},
\texttt{ObjectiveDirection.Maximize}.

\subsection{\texttt{PropagationTrigger}}
Tato struktura obsahuje nějakou jednotlivou událost, na kterou mohou reagovat
jednotlivé CSP podmínky. Enumerace \texttt{PropagationTrigger.Type} obsahuje
typy \texttt{Restrict} a \texttt{Assign}. \texttt{Restrict} znamená, že
z definičního oboru nějaké proměnné zmizela nějaká hodnota, a lze vyrobit
statickou metodou \texttt{Restrict}, jejíž parametry jsou \texttt{Variable} a
\texttt{int}. Obdobně \texttt{Assign}
znamená, že do dané proměnné byla přiřazena hodnota (resp. definiční obor této
proměnné se zmenšil na jeden prvek). Taktéž jde vyrobit statickou metodou
\texttt{Assign(Variable, int)}. Struktura přetěžuje \texttt{ToString()}.

\subsection{\texttt{Problem}}
Třída \texttt{Problem} zveřejňuje přes vlastnosti \texttt{Variables} a
\texttt{Constrains} rozhraní k manipulaci s proměnnými a
podmínkami problému. Tato rozhraní mají jsou definována interfacy
\texttt{Problem.IVariables} a \texttt{Problem.IConstrains}. Konkrétní
implementace těchto rozhraní jsou třídy \texttt{Problem.Variables} a
\texttt{Problem.Constrains}.

Všechny proměnné v rámci jednoho problému mají unikátní identifikátory.
Privátní metoda \texttt{GenerateVariableName()} třídy \texttt{Problem} generuje
tyto identifikátory (nezadal-li je uživatel třídy).
Třída \texttt{Problem} interně ukládá proměnné v \texttt{List<Variable>
variables} a podmínky v \texttt{List<Constrains.IConstrain> constrains}.

Rozhraní \texttt{IVariables} poskytuje následující funkce:
\begin{itemize}
\item \texttt{Variable[] AddIntegers(int count, int min, int max, Func<int,
	string> naming = null)}
\item \texttt{Variable AddInteger(int min, int max, string name = null)}
\item \texttt{Variable AddInteger(ValueRange, string name = null)}
\item \texttt{Variable AddBoolean(string name = null)}
\item \texttt{Variable AddBooleans(int count, Func<int, string> naming = null)}
\end{itemize}

Rozhraní \texttt{IConstrains} poskytuje následující funkce:
\begin{itemize}
\item \texttt{void Add(params IConstrain[])}
\item \texttt{void Add(IEnumerable<IConstrain>)}
\item \texttt{void Remove(params IConstrain[])}
\item \texttt{void Remove(IEnumerable<IConstrain>)}
\end{itemize}

Kromě těchto rozhraní dále třída \texttt{Problem} má tyto metody:
\begin{itemize}
\item \texttt{IVariableAssignment CreateEmptyAssignment()}:
	vytvoří přiřazení hodnot, které každé proměnné pouze přiřadí její
	standardní doménu.
\item \texttt{List<IConstrain> AllConstrains()}
\item \texttt{IExternalEnumerator<Variable> EnumerateVariables()}
\item \texttt{void SetObjective(Variable, ObjectiveDirection)}:
	nastaví, že cílem problému je optimalizovat danou proměnnou v daném
	směru.
	Interně se toto ukládá ve veřejných vlastnostech
	\texttt{ObjectiveVariable}, resp. \texttt{Direction}. Je-li
	\texttt{ObjectiveVariable == null}, neprovádí se optimalizace. Toto také
	lze zjistit getterem \texttt{public bool IsOptimization \{ get; \}}.
\end{itemize}

\subsection{\texttt{Solver}}
\texttt{Solver} je třída, která pro daný \texttt{Problem} hledá řešení -- buď
libovolné, nebo optimální ve smyslu
\texttt{ObjectiveVariable}+\texttt{Direction}.

Veřejné rozhraní se skládá jenom z metod \texttt{Solve}, \texttt{SolveParallel}
a \texttt{SolveSerial}. Všechny mají stejnou sémantiku volání: první parametr je
řešená CSP instance (\texttt{Problem}) a druhý output parametr je
\texttt{IVariableAssignment} -- do tohoto parametru se v případě úspěchu uloží
nalezené řešení. Vracejí boolean, který říká, jestli se povedlo najít řešení.
\texttt{Solve} deleguje volání na \texttt{SolveParallel}.

Interně solver používá instance třídy \texttt{SolutionStep},
která reprezentuje průběh řešení. Protected metoda
\texttt{bool SearchSerial(SolutionState, out IVariableAssignment, CancellationToken)}
provádí DFS nad daným \texttt{SolutionState} a pokud je
\texttt{CancellationToken} cancelnutý, ukončí hledání a vrátí neúspěch.
Paralelní hledání realizuje metoda \texttt{SearchAsync}.
Vrací \texttt{Task<IVariableAssignment>}, jehož výsledek je buď \texttt{null},
nebo splňující řešení. Její parametry jsou \texttt{SolutionState}, \texttt{int
depth} a \texttt{CancellationToken}.
\texttt{depth} se používá k paralelizaci pouze na několika (3) prvních
proměnných, aby režie \texttt{Task}ů a vláken nebrala více času než skutečné
hledání řešení (ve větší hloubce se přejde k \texttt{SearchSerial}).
\texttt{CancellationToken}y předávají mezi jednotlivými větvemi výpočtu
informaci o tom, že některá větev úspěšně skončila, takže ostatní větve můžou
skončit s výpočty.

Samotné metody \texttt{SearchSerial} a \texttt{SearchAsync} neumí diskrétní
optimalizaci. Optimalizaci provádí protected metoda
\texttt{bool Optimize(Problem, out IVariableAssigment, Func<Problem,
IVariableAssignment>)}. Když metody \texttt{SolveParallel} nebo \texttt{SolveSerial}
dostanou problém s optimalizací, delegují jej na metodu \texttt{Optimize}, které
jako třetí argument "předají samy sebe".

Diskrétní optimalizace funguje metodou branch-and-bound. Nejdříve se najde
libovolné řešení, a na základě jeho hodnoty optimalizované proměnné se přidá
do CSP podmínka, že hodnota optimalizované proměnné musí být větší, resp. nižší
než zatím nejlepší známá. Informace o úspěšně nalezených řešeních a
neřešitelných instancích nakonec dají půlením intervalu nejlepší splnitelnou
hodnotu optimalizované proměnné.

Metoda \texttt{Optimize} tedy do vstupní instance \texttt{Problem} přidává
podmínky odpovídající známým odhadům na hodnotu optimalizované proměnné a
nechává neoptimalizující solver najít libovolné řešení.

\subsection{\texttt{SolutionState}}
Třída \texttt{SolutionState} je stav řešení CSP problému. Kromě přiřazení hodnot
proměnným (\texttt{IVariableAssignment assignment}) také obsahuje množinu doposud nesplněných podmínek
(\texttt{List<IConstrain> Unsolved}) a externí enumerátor proměnné, jejíž
hodnota se bude přiřazovat v dalším kroku, a hodnoty, která se bude zkoušet (\texttt{IExternalEnumerator<Variable>
VariableChoice},\\ \texttt{IExternalEnumerator<int> ValueChoice}).
Aby bylo možné provádět backtracking přes \texttt{SolutionState}, implementuje
tato třída rozhraní \texttt{IBacktrackable<SolutionState>} (vnitřní implementace
používá \texttt{Stack}).

Veřejné rozhraní dále obsahuje tyto metody a vlastnosti:
\begin{itemize}
\item \texttt{bool Success \{ get; \}}: vrací, byly-li už všechny
	podmínky vyřešeny.
\item \texttt{void MarkConstrainSolved(IConstrain)}: označí danou podmínku za
	splněnou
\item \texttt{bool PropagateInitial()}: projde všechny proměnné, které mají
	přiřazenou hodnotu, a zavoláním \texttt{PropagateTriggers} tuto informaci
	rozpropaguje přes všechny podmínky.
\item \texttt{void SetValueChoice()}: nastaví počáteční hodnotu
	\texttt{ValueChoice} (na externí enumerátor přípustných hodnot
	\texttt{VariableChoice})
\item \texttt{void Assign(Variable, int)}, \texttt{void Restrict(Variable,
	int)}: změní daným způsobem doménu proměnné
\item \texttt{bool NextValueChoice()}: pokusí se nastavit další přípustnou
	hodnotu \texttt{ValueChoice}.
\item \texttt{bool Progress()}: přiřadí do \texttt{VariableChoice} hodnotu
	\texttt{ValueChoice} a provede propagaci této informace přes
	\texttt{PropagateTriggers}. Pokud propagace skončí neúspěchem, vrátí
	\texttt{false}.
\item \texttt{void Dump()}: vypíše na standardní výstup popis tohoto stavu
	řešení.
\end{itemize}

Privátní metoda \texttt{bool PropagateTriggers(IEnumerable<PropagationTrigger>)}
oznámí všem podmínkám uvedené změny v řešení (momentálně pouze typu
\texttt{Restrict} nebo \texttt{Assign}, ale pro praktické použití se mohou hodit
i jiné informace -- například změny minim a maxim, průnik s intervalem, atd.).
Pokud nějaká podmínka z těchto informací odvodí nějaké další změny, tak se tyto
změny ze všech podmínek znova zpropagují. Toto se opakuje, dokud nějaká podmínka
odvozuje nové informace.
Tato metoda vrací \texttt{true}, pokud propagace proběhla úspěšně. Pokud nějaká
podmínka oznámila, že už nebude nikdy splnitelná, vrátí se \texttt{false}, což
znamená, že se musí zaříznout DFS větev.

Privátní metoda \texttt{ResolveFullyInstantiatedConstrains} najde všechny
podmínky, jejichž závislosti už mají přiřazeny hodnoty. Narazí-li na "plně
instanciovanou podmínku", která není splněna, vrátí \texttt{false}. Jinak všechny
splněné "plně instanciované podmínky" označí za splněné a vrátí \texttt{true}.
Používá se interně při propagaci k zajištění, že podmínky, které neprovádí
žádnou vlastní propagaci, budou okamžikem dosazení do všech závislostí
"odstraněny z agendy".

\subsection{\texttt{CompulsiveSkinPicking.AlgebraicExpression}}
Namespace \texttt{CompulsiveSkinPicking.AlgebraicExpression} obsahuje komponenty
umožňující pro daný problém vyrobit proměnnou, která je vázána na výsledek
nějakého výpočtu.
Pokud například máme proměnné S, E, N a D obsahující číslice, můžeme
následujícím způsobem vyrobit proměnnou reprezentující hodnotu SEND:
\begin{lstlisting}
Variable SEND =
  (S * 1000 + E * 100 + N * 10 + D).Build(problem);
\end{lstlisting}

Z výrazu se sestrojí strom skládající se z tříd dědících od abstraktní třídy
\texttt{Node}. Mezi její potomky patří třídy \texttt{VariableNode},
\texttt{ConstantNode}, \texttt{UnaryNode} (na unární operace) a
\texttt{BinaryNode} (na binární operace). Metoda \texttt{Build} zavolaná na
třídě \texttt{Node} zavolá abstraktního visitora \texttt{ExpressorVisitor},
který v předaném problému vytvoří hierarchii pomocných proměnných svázaných
podmínkami, odpovídající danému výrazu.

Instance \texttt{UnaryNode} a \texttt{BinaryNode} si pamatují svůj "typ" --
například enumerate \texttt{BinaryNode.Type} obsahuje prvky \texttt{Plus},
\texttt{Minus}, \texttt{Multiply}, \texttt{Divide}, \texttt{Modulo},
\texttt{And}, \texttt{Or}, \texttt{Xor}, \texttt{Implies}, \texttt{GreaterThan}
a \texttt{GreaterThanOrEqualTo}. Každá hodnota se přeloží na jeden druh podmínky
ve statické třídě \texttt{Constrain}.

\subsection{\texttt{Constrain}}
Následující podmínky je možno zkontruovat pomocí statické třídy \texttt{Constrain},
která poskytuje tenké rozhraní nad konstruktory tříd z namespacu \\
\texttt{CompulsiveSkinPicking.Constrains}:
\begin{itemize}
\item
	\texttt{NotEqual(X,Y)}: hodnoty proměnných musí být různé
\item
	\texttt{Equal(X,Y)}: hodnoty proměnných musí být shodné
\item
	\texttt{Plus(X,Y,Z)}: součet hodnot proměnných X a Y musí být hodnota Z
\item
	\texttt{Minus(X,Y,Z)}: analogické k \texttt{Plus}
\item
	\texttt{Multiply(X,Y,Z)}: analogické k \texttt{Plus}
\item
	\texttt{Divide(X,Y,Z)}: analogické k \texttt{Plus}. Jedná se o
	celočíselné dělení -- například $15 / 6 = 2$.
\item
	\texttt{Modulo(X,Y,Z)}: analogické k \texttt{Plus}. Jde o zbytek po
	celočíselném dělení -- například $15 \% 6 = 3$.
\item
	\texttt{VariableNot(X,Y)}: pravdivost proměnné X musí být opak
	logické hodnoty proměnné Y. Pravdivostí se pro účely této podmínky
	a všech následujících rozumí nenulovost. (Toto umožňuje kombinovat
	booleovské a celočíselné proměnné.)
\item
	\texttt{VariableAnd(X,Y,Z)}: pravdivost Z musí být pravdivost X $\wedge$
	pravdivost Y
\item
	\texttt{VariableOr(X,Y,Z)}: analogicky, $\vee$
\item
	\texttt{VariableXor(X,Y,Z)}: analogicky, $\oplus$
\item
	\texttt{VariableImplies(X,Y,Z)}: analogicky, $\rightarrow$
\item
	\texttt{VariableGreaterThan(X,Y,Z)}: proměnná Z je pravdivá právě když
	hodnota X je větší, než hodnota Y.
\item
	\texttt{VariableGreaterThanOrEqualTo(X,Y,Z)}: analogicky, $\geq$
\item
	\texttt{GreaterThan(X,Y)}: hodnota X musí být větší než hodnota Y
\item
	\texttt{GreaterThanOrEqualTo(X,Y,Z)}: analogicky, $\geq$
\item
	\texttt{AllDifferent(X,Y,\ldots)}: všechny předané proměnné musí mít
	různé hodnoty. Přijímá jak pole proměnných, tak proměnné uvedené jako
	jednotlivé parametry.
\item
	\texttt{Truth(X)}: proměnná X má pravdivou hodnotu
\item
	\texttt{Relational(r,X[])}: na předaném poli proměnných platí daná
	relace. Relace je funkce, která dostane jako parametry hodnoty
	jednotlivých proměnných (\texttt{int}) a vrátí, je-li splněna.
	Příklad použití:
\begin{lstlisting}
problem.Constrains.Add(Constrain.Relational(
	(x) => (x[0] + x[1] == x[2]) || (x[0] - x[2] == x[1]),
	new Variable[] { A, B, C }
));
\end{lstlisting}
	Proměnné lze uvést polem nebo jednotlivými argumenty.
\item
	\texttt{BinaryFunctional(f,A,B,Y)}: $Y=f(A,B)$. Příklad použití:
\begin{lstlisting}
problem.Constrains.Add(Constrain.BinaryFunctional(
	(a,b) => Math.Abs(a - b),
	X, Y, Z
)); // |X-Y|=Z
\end{lstlisting}
\end{itemize}

\section{Testy}

Knihovna obsahuje několik testů svých funkcí.
Tyto testy jsou uloženy v namespacu \texttt{CompulsiveSkinPicking.Tests}.

\begin{itemize}
\item
	\texttt{NQueens} je klasický test CSP solverů. Snaží se rozmístit $N$
	královen na šachovnici $N\times N$ tak, aby se navzájem neohrožovaly.
\item
	\texttt{SAT} ukazuje možnost řešit instance SAT přes CSP solver.
\item
	\texttt{SendMoreMoney} a \texttt{SendMoreMoneyMaximize} ukazují řešení
	algebrogramu $SEND+MORE=MONEY$, kde druhý test se snaží maximalizovat
	číslo \texttt{MONEY}.
\item
	\texttt{ThreeColorsInGraph} ukazuje barvení grafu pomocí CSP.
\item
	\texttt{TwoTwoFour} je další algebrogram: $TWO+TWO=FOUR$.
\item
	Test \texttt{LargeAlgebrogram} řeší o něco složitější problém, než
	\texttt{SendMoreMoney}: $(WE)(WANT)+SOME=MORE+MONEY+PLEASE/3$,
	kde $S$,$E$,$N$,$M$,$O$,$R$,$Y$,$W$,$T$ jsou různé a $S$,$A$,$P$,$L$
	jsou různé.
\end{itemize}

\end{document}
