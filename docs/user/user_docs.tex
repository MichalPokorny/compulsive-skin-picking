\documentclass[a4paper]{article}
% \usepackage{fullpage}
\usepackage[utf8]{inputenc}
\usepackage[nottoc,numbib]{tocbibind}
% \usepackage{amssymb}
% \usepackage{amsmath}
\usepackage[czech]{babel}
\usepackage{listings}
\usepackage{todonotes}
\usepackage{fancyhdr}
\usepackage{listings}
\usepackage{wrapfig}
\usepackage{caption}
\usepackage{subcaption}
\usepackage{placeins}

% \definecolor{string}{rgb}{0.7,0.0,0.0}
% \definecolor{comment}{rgb}{0.13,0.54,0.13}
% \definecolor{keyword}{rgb}{0.0,0.0,1.0}
% \lstset{
% % numbers=left,
%   language=Ruby,
%   frame=single,
%   tabsize=2,
%   basicstyle=\footnotesize\ttfamily,
%   keywordstyle=\color{keyword}\textbf,
%   commentstyle=\color{comment}\textit,
%   stringstyle=\color{string}
% }
% \pgfdeclarelayer{background}
% \pgfdeclarelayer{foreground}
% \pgfsetlayers{background,main,foreground}

\begin{document}
\title{Uživatelská dokumentace \\ CompulsiveSkinPicking - C\# CSP solver}
\author{Michael Pokorný}
\date{13.7.2014}

\maketitle

\tableofcontents

\section{Úvod}
CSP (\textit{Constrain Satisfaction Problem}) je klasický příklad
NP-kompletního problému.
Představme si množinu \textit{proměnných} $V$, kde proměnná
$V_i$ může nabývat pouze hodnot z $D(V_i)$. Dále omezme hodnoty těchto proměnných
\textit{podmínkami} $C$. Každá podmínka $C_i$ je $n_i$-ární relace nad definičními
obory $n_i$ proměnných:
$$C_i \subseteq D(V_{v_1})\times \cdots \times D(V_{v_{n_i}})$$
Řešení CSP problému je takové přiřazení hodnot z $D(V_i)$ každé proměnné
$V_i$, které splňuje všechny tyto podmínky. NP-kompletnost lze dokázat například
triviálním převodem SAT na CSP.

CSP se často používá jako alternativní reprezentace jiných těžkých problémů v
umělé inteligenci. Při vhodném překladu z původního problému na CSP lze výhodně
využít existující techniky pro CSP řešiče. Na CSP se například často převádí
plánovací a rozvrhovací problémy.

Účel této knihovny je paralelně řešit zadanou instanci CSP.

\section{Reprezentace problému}
Nejdříve je potřeba vytvořit reprezentaci daného problému.
K tomu slouží třída \texttt{Problem}:
\begin{lstlisting}
using CompulsiveSkinPicking;
...
Problem problem = new Problem();
\end{lstlisting}

Problém se skládá z proměnných a podmínek.
Nezáleží na pořadí, v jakém se přidávají -- pouze musí být každá podmínka
přidána po všech proměnných, na kterých závisí.

\subsection{Proměnné}
Knihovna CompulsiveSkinPicking umožňuje pouze používat proměnné s konečnými
diskrétními doménami. Proměnná se reprezentují instancemi třídy
\texttt{Variable} a lze je přidávat voláním metod vlastnosti \texttt{Variables}
třídy \texttt{Problem}. Takto lze přidat buď celočíselné proměnné, nebo
booleovské proměnné. Booleovské proměnné jsou pouze speciální případ, lišící se
standardní doménou $\{0;1\}$:
\begin{lstlisting}
// 0 <= X <= 10
Variable x = problem.Variables.AddInteger(0, 10);
// Volitelne jde nastavit jmeno.
Variable x2 = problem.Variables.AddInteger(0, 10, "X2");

// 3 promenne, -5 <= X < -3
Variable[] abc = problem.Variables.AddIndeger(3, -5, -3);
// Lze nastavit pojmenovavaci konvenci. Predane funkci
// se predavaji indexy 0, 1, 2 pro pojmenovani kazde
// promenne.
Variable[] prom123 =
  problem.Variables.AddIndeger(3, -5, -3, x =>
    string.Format("Prom{0}", x+1)
  );

Variable x = problem.Variables.AddBoolean();
Variable xs = problem.Variables.AddBooleans(10);
\end{lstlisting}

\subsection{Podmínky}
Pro přidávání podmínek slouží vlastnost \texttt{Constrains} třídy
\texttt{Problem}. Ke konstrukci podmínek se dá použít statická třída
\texttt{Constrain}. Je možné také zkonstruovat vlastní podmínky (což popisuje
podrobněji programátorská dokumentace).

Následující podmínky je možno zkontruovat pomocí třídy \texttt{Constrains}:
\begin{itemize}
\item
	\texttt{NotEqual(X,Y)}: hodnoty proměnných musí být různé
\item
	\texttt{Equal(X,Y)}: hodnoty proměnných musí být shodné
\item
	\texttt{Plus(X,Y,Z)}: součet hodnot proměnných X a Y musí být hodnota Z
\item
	\texttt{Minus(X,Y,Z)}: analogické k \texttt{Plus}
\item
	\texttt{Multiply(X,Y,Z)}: analogické k \texttt{Plus}
\item
	\texttt{Divide(X,Y,Z)}: analogické k \texttt{Plus}. Jedná se o
	celočíselné dělení -- například $15 / 6 = 2$.
\item
	\texttt{Modulo(X,Y,Z)}: analogické k \texttt{Plus}. Jde o zbytek po
	celočíselném dělení -- například $15 \% 6 = 3$.
\item
	\texttt{VariableNot(X,Y)}: pravdivost proměnné X musí být opak
	logické hodnoty proměnné Y. Pravdivostí se pro účely této podmínky
	a všech následujících rozumí nenulovost. (Toto umožňuje kombinovat
	booleovské a celočíselné proměnné.)
\item
	\texttt{VariableAnd(X,Y,Z)}: pravdivost Z musí být pravdivost X $\wedge$
	pravdivost Y
\item
	\texttt{VariableOr(X,Y,Z)}: analogicky, $\vee$
\item
	\texttt{VariableXor(X,Y,Z)}: analogicky, $\oplus$
\item
	\texttt{VariableImplies(X,Y,Z)}: analogicky, $\rightarrow$
\item
	\texttt{VariableGreaterThan(X,Y,Z)}: proměnná Z je pravdivá právě když
	hodnota X je větší, než hodnota Y.
\item
	\texttt{VariableGreaterThanOrEqualTo(X,Y,Z)}: analogicky, $\geq$
\item
	\texttt{GreaterThan(X,Y)}: hodnota X musí být větší než hodnota Y
\item
	\texttt{GreaterThanOrEqualTo(X,Y,Z)}: analogicky, $\geq$
\item
	\texttt{AllDifferent(X,Y,\ldots)}: všechny předané proměnné musí mít
	různé hodnoty. Přijímá jak pole proměnných, tak proměnné uvedené jako
	jednotlivé parametry.
\item
	\texttt{Truth(X)}: proměnná X má pravdivou hodnotu
\item
	\texttt{Relational(r,X[])}: na předaném poli proměnných platí daná
	relace. Relace je funkce, která dostane jako parametry hodnoty
	jednotlivých proměnných (\texttt{int}) a vrátí, je-li splněna.
	Příklad použití:
\begin{lstlisting}
problem.Constrains.Add(Constrain.Relational(
	(x) => (x[0] + x[1] == x[2]) || (x[0] - x[2] == x[1]),
	new Variable[] { A, B, C }
));
\end{lstlisting}
	Proměnné lze uvést polem nebo jednotlivými argumenty.
\item
	\texttt{BinaryFunctional(f,A,B,Y)}: $Y=f(A,B)$. Příklad použití:
\begin{lstlisting}
problem.Constrains.Add(Constrain.BinaryFunctional(
	(a,b) => Math.Abs(a - b),
	X, Y, Z
)); // |X-Y|=Z
\end{lstlisting}
\end{itemize}

\subsection{Výrazové proměnné}
Zatím uvedené podmínky nedávají dost pohodlí ke konstrukci složitějších
algebraických omezení. K tomu lze využít výrazové proměnné.
Jedná se o mechanismus, který na základě matematického výrazu nad proměnnými
vyrobí novou proměnnou reprezentující jeho hodnotu.

Pokud například máme proměnné S, E, N a D obsahující číslice, můžeme
následujícím způsobem vyrobit proměnnou reprezentující hodnotu SEND:
\begin{lstlisting}
Variable SEND =
  (S * 1000 + E * 100 + N * 10 + D).Build(problem);
\end{lstlisting}

Tento mechanismus je potřeba použít pro vytvoření proměnné v místě,
kam bychom chtěli předat konstantu:
\begin{lstlisting}
problem.Constrains.Add(
  Constrain.Plus(X, Y, 5.Build(problem))
);
\end{lstlisting}

Mezi podporované operátory patří závorky, \texttt{+}, \texttt{-}, \texttt{*},
\texttt{/}, \texttt{\%}, \texttt{\&}, \texttt{|}, \texttt{\^}, \texttt{!}.
Kromě toho lze jako "pseudooperátory" používat statické metody
\texttt{GreaterThan}, \texttt{GreaterThanOrEqualTo}, \texttt{LessThan},
\texttt{LessThanOrEqualTo} třídy
\texttt{CompulsiveSkinPicking.AlgebraicExpression.Node}. Tyto metody vrací
booleovskou proměnnou, jejíž pravdivost záleží na splnění dané relace.

\subsection{Optimalizace}
Některé úlohy kromě splnění podmínek také vyžadují minimalizaci nebo
maximalizaci hodnoty některé proměnné. Rozvrhování například může vyžadovat
co nejkratší dobu vykonávání rozvrhu nebo co nejméně použitých zdrojů.

Proměnnou k optimalizaci a směr optimalizace lze nastavit zavoláním metody
\texttt{SetObjective} třídy \texttt{Problem}:
\begin{lstlisting}
problem.SetObjective(RESOURCES_USED, ObjectiveDirection.Minimize);
\end{lstlisting}

Jeden problém může optimalizovat nejvýše jednu proměnnou.

\section{Volání solveru}
Solver je komponenta řešící CSP problém. Knihovna CompulsiveSkinPicking obsahuje
pouze jednu implementaci:
\begin{lstlisting}
Solver solver = new Solver();
\end{lstlisting}
Když je hotové zadání problému, stačí k jeho vyřešení zavolat na solveru metody
\texttt{Solve}. Její první parametr je problém a její druhý výstupní parametr je
proměnná, do které bude uloženo řešení (typu
\texttt{CompulsiveSkinPicking.IVariableAssignment}). Tato metoda vrátí boolean, který značí,
bylo-li řešení úspěšné:
\begin{lstlisting}
IVariableAssignment solution;
if (!solver.Solve(problem, out solution)) {
  Console.WriteLine("No solution exists :(");
  Environment.Exit(1);
}
Console.WriteLine("Solution found");
\end{lstlisting}

Nalezené řešení lze zkoumat indexerem, jehož parametr je proměnná:
\begin{lstlisting}
Console.WriteLine("Resources used: {0}",
  solution[RESOURCES_USED].Value
);
\end{lstlisting}

\end{document}
